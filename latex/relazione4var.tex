\documentclass{article}

\title{Relazione 4: Misura del passo di un reticolo di diffrazione e di lunghezze d'onda}
\date{}

\begin{document} \maketitle	
	
Nella presente esperienza si vuole verificare la legge del reticolo di diffrazione $\sin{\theta}=m\frac{\lambda}{p}$, che lega la posizione angolare $\theta$ dei massimi principali della figura di diffrazione rispetto al massimo centrale al rapporto tra la lunghezza d'onda $\lambda$ e il passo del reticolo $p$.
	
A tal fine, si intende utilizzare prima una lampada spettrale al sodio per misurare il passo del reticolo, avvalendosi di una riga spettrale con lunghezza d'onda nota; con tale misura, successivamente, sarà possibile stimare le lunghezze d'onda delle principali righe di emissione di una lampada al cadmio.

L'apparato sperimentale, mostrato in Figura 1, è costituito da uno spettroscopio, le cui principali caratteristiche costruttive sono di seguito riportate.
Lo strumento è dotato di un piattello centrale, capace di ruotare attorno ad un asse fisso; su esso andrà collocato il reticolo.
Solidale alla base dello strumento, vi è un cannocchiale collimatore, il quale è dotato di un meccanismo per modificare le dimensioni della fenditura di ingresso; di fronte all'imbocco del cannocchiale saranno posizionate le lampade. 
Infine, un cannocchiale di mira è libero di ruotare attorno all'asse centrale dello strumento. Il cannocchiale possiede un meccanismo per regolarne la messa a fuoco ed uno per modificarne l'inclinazione; 
inoltre, con una manopola è possibile bloccarne la posizione e con un'altra è possibile eseguire spostamenti angolari di qualche grado primo mentre è bloccato.
Sul suo oculare è presente un crocifilo; durante le misure, la messa a fuoco del mirino andrà regolata in modo che le righe spettrali risultino abbastanza sottili da sovrapporsi al filo verticale. In questo modo, si aumenterà la precisione dello strumento.   

Le misure di angoli saranno realizzare mediante il goniometro affisso alla base dello spettroscopio, con sensibilità $S=0.5^{\circ} / div$, ed il nonio solidale al cannocchiale di mira, che rende possibile eseguire misure con errore di sensibilità pari a
$\Delta S = 1'$.
Le lampade utilizzate, dopo essere state accese, richiedono qualche minuto per raggiungere la temperatura di lavoro.

Si pone la lampada al Na accesa di fronte al collimatore.
Tutte le misure successive sono realizzate centrando il fascio luminoso di cui si vuole misurare la posizione angolare con il crocefilo; quindi si blocca il mirino e se ne legge la posizione angolare sul goniometro.

Prima di porre il reticolo sul piattello, si allinea il cannocchiale con il fascio uscente dal collimatore e se ne registra la posizione angolare $\Theta_0=83^{\circ}20'\pm1'$. Quindi, si colloca il reticolo al centro del piattello, perpendicolarmente al fascio di luce proveniente dal collimatore. Prima di procedere alla misura di $p$, occorre assicurarsi che il reticolo sia correttamente posizionato, al fine di ridurre eventuali errori sistematici.
 
Un modo di fare ciò è verificare che la figura di diffrazione generata risulti simmetrica rispetto al massimo centrale entro un errore $\varepsilon = 5'$. Innazitutto, si misura la posizione angolare del massimo centrale $\Theta_1 = 83^\circ 24' \pm 1'$ e si osserva che non è compatibile con $\Theta_0$.
Quindi, scegliendo come riga spettrale rispetto a cui eseguire le misure quella del caratteristico "doppietto del sodio" corrispondente a $\lambda=589,6nm$ , si misurano le posizioni dei massimi principali di ordine +2  e -2, rispettivamente pari a $\Theta_s = 128^\circ 44' \pm 1'$
e $\Theta_d = 37^\circ 10' \pm 1'$. Si osserva che le loro distanze dal massimo centrale, rispettivamente $\theta_s = |\Theta_s - \Theta_1| = 45^\circ20' \pm 2'$ e $\theta_d = |\Theta_d - \Theta_1| = 46^\circ 14' \pm 2'$, differiscono per $\Delta\theta = |\theta_s - \theta_d| = 54' \pm 4'$. Pertanto, il reticolo di diffrazione non risulta sufficientemente ben allineato. Per ridurre tale differenza, si sposta il cannocchiale di $\Delta\theta/2$ a sinistra del massimo di ordine -2, collocandolo nella posizione in cui ci si
aspetterebbe di trovare la riga se il reticolo fosse ben posizionato. Successivamente, si ruota il reticolo fino ad allineare la riga con il crocifilo e si misurano nuovamente $\Theta_1 = 83^\circ 25' \pm 1'$, $\Theta_s = 129^\circ 9' \pm 1'$ e $\Theta_d = 37^\circ 34' \pm 1'$; da tali misure risulta $\Delta\theta
= 4' \pm 4'$, compatibile con zero entro gli errori sperimentali. A questo punto, si può ritenere il reticolo sufficientemente ben posizionato.

Al fine di stimare il passo $p$ del reticolo, restano da misurare le posizioni angolari dei massimi principali di ordine 1 e -1.
Tali misure e le precedenti (ordini -2, 0, 2) sono riportate in Tabella 2, dove, in corrispondenza dell'ordine $m$, si elencano le misure di $\Theta_m$, $\theta_m = \Theta_m - \Theta_0$ (ora con $\Theta_0$ si indica la posizione assoluta del massimo centrale) e $\sin \theta_m$.
Sul Grafico 1 è rappresentato l'andamento di $\sin \theta_m$ in funzione dell'ordine $m$. Tramite una regressione lineare è possibile stimare la pendenza della retta di best fit $\sin \theta_m = Am + B$ e verificare che l'intercetta risulti compatibile con zero. Dunque, è possibile stimare il passo del reticolo come $p=\frac{\lambda}{A}$.
Dalla regressione lineare effettuata sui dati in Tabella 2, risultano $A=0,35834\pm0,00013$ e $B=0,0000\pm0,0002$, da cui $p=(1,6454\pm0,0006)\mu m$. Se ne conclude che i dati speramentali sono in accordo con la legge teorica del reticolo e che la stima di $p$ è consistente con il valore nominale fornito dal costruttore.

Volendo ora determinare le lunghezze d'onda dello spettro di emissione del cadmio, si sostituisce la lampada al Na con la lampada al Cd e si ripete la misura delle posizioni angolari dei massimi principali (incluso il massimo centrale) per ciascun colore dello spettro visibile.
In particolare, si distinguono quattro colori: indaco, azzurro, verde e rosso. Nelle Tabelle 3 - 6 sono riportate le misure di $m$, $\theta_m$, $\sin \theta_m$ rispettivamente per ciascuno dei suddetti colori.
Parimenti, nei Grafici 2 - 5 sono riportati gli andamenti di $\sin \theta_m$ in funzione dell'ordine $m$, sui quali si eseguono altrettante regressioni lineari.
Di seguito si riportano i risultati (parametri $A$ e $B$) delle suddette regressioni lineari e la stima corrispondente della lunghezza d'onda, ottenuta come $\lambda_{colore}= A_{colore} p$, con il valore di $p$ precedentemente stimato:
$\lambda_{indaco} = (467.8 \pm 0.2)nm$;
$\lambda_{azzurro} = (479.8 \pm 0.2)nm$;
$\lambda_{verde} = (508.5 \pm 0.2)nm$;
$\lambda_{rosso} = (643.5 \pm 0.3)nm$.

[Si è inoltre osservato che questi valori risultano compatibili entro gli errori con quelli riportati dal National Institute of Standards and Technology nell'Atomic Spectra Database: $\lambda^'_{indaco} = 467.81493 nm$, $\lambda^'_{azzurro} = 479.99123 nm$, $\lambda^'_{verde} = 508.58214$, $\lambda^'_{rosso} = 643.84695$.]

In conclusione, si osserva un buon accordo tra i dati sperimentali relativi a ciascun colore dello spettro visibile del cadmio con la legge teorica del reticolo. Inoltre, la stima di ciascuna lunghezza d'onda è compatibile con il valore atteso per essa.



 
	
	
\end{document}
