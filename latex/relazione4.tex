\documentclass{article}

\title{Relazione 4: Misura del passo di un reticolo di diffrazione
	e di lunghezze d'onda}

\date{}

\begin{document}
	\maketitle
	
	
	In questa esperienza si utilizza un reticolo di diffrazione per analizzare le righe spettrali di lampade al Na e al Cd.  [Da qualche parte all'inizio va messa la formula del reticolo]
	
	Si intende utilizzare prima la lampada al Na per misurare il passo del reticolo avvalendosi di una riga spettrale con lunghezza d'onda $\lambda$ nota, per poi osservare lo spettro della lampada al Cd e ricavarne le lunghezze d'onda delle principali righe spettrali.
	
	L'apparato sperimentale, [mostrato in Figura 1], è costituito da un piattello rotante intorno a un asse fisso con un collimatore ed un cannocchiale ad esso solidali ma capaci di ruotare indipendentemente da esso. Sull'asse è inoltre posto un portaoggetti fisso, sul quale si pone il reticolo di diffrazione.
	
	Sul piattello è affisso un goniometro con sensibilità $S=0.5^{\circ} / div$ e al cannocchiale è affisso un indice dotato di un nonio che, utilizzato in combinazione con il goniometro, rende possibile prendere misure con errore di sensibilità pari a $\Delta S = 1'$.

 Dopo aver acceso la lampada al Na e averla posta di fronte al collimatore, è necessario aspettare qualche minuto per assicurarsi che abbia raggiunto la temperatura di lavoro. Poi si riduce al l'apertura della fenditura del collimatore e si osserva la sua immagine attraverso il cannocchiale: in questo modo ci si può assicurare che l'immagine sia nitida e abbia le minime dimensioni trasversali, ma anche che sia sufficientemente luminosa e visibile.

Prima di porre il reticolo sul piattello, si allinea il cannocchiale con il collimatore e se ne registra la posizione angolare $\Theta_0$. Si procede quindi a collocare il reticolo al centro del piattello, perpendicolarmente al fascio di luce proveniente dal collimatore. Prima di continuare, si ci assicura che il reticolo sia correttamente posizionato entro un errore di $\varepsilon = 5'$, al fine di evitare errori sistematici.
	
 [MATTEO]



 
	
	
\end{document}
