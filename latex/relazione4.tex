\documentclass{article}

\title{Relazione 4: Misura del passo di un reticolo di diffrazione
	e di lunghezze d'onda}

\date{}

\begin{document}
	\maketitle
	
	
	In questa esperienza si utilizza un reticolo di diffrazione per analizzare le righe spettrali di lampade al Na e al Cd.  [Da qualche parte all'inizio va messa la formula del reticolo]
	
	Si intende utilizzare prima la lampada al Na per misurare il passo del reticolo avvalendosi di una riga spettrale con lunghezza d'onda $\lambda$ nota, per poi osservare lo spettro della lampada al Cd e ricavarne le lunghezze d'onda delle principali righe spettrali.
	
	L'apparato sperimentale, [mostrato in Figura 1], è costituito da un piattello rotante intorno a un asse fisso con un collimatore ed un cannocchiale ad esso solidali ma capaci di ruotare indipendentemente da esso. Sull'asse è inoltre posto un portaoggetti fisso, sul quale si pone il reticolo di diffrazione.
	
	Sul piattello è affisso un goniometro con sensibilità $S=0.5^{\circ} / div$ e al cannocchiale è affisso un indice dotato di un nonio che, utilizzato in combinazione con il goniometro, rende possibile prendere misure con errore di sensibilità pari a $\Delta S = 1'$.
	
	[Qualcosa sul collocare la lampada attaccata al collimatore, stringere il piu possibile la fenditura prima di usare il reticolo, aspettare che la lampada si riscaldi e altri accorgimenti?]

	Per prima cosa, è necessario assicurarsi che il fascio collimato proveniente dalla lampada incida perpendicolarmente sul reticolo di diffrazione al fine di evitare grossolani errori sistematici. 
	
	
\end{document}