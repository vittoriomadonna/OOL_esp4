\documentclass{article}

\title{Relazione 4: Misura del passo di un reticolo di diffrazione
	e di lunghezze d'onda}

\date{}

\begin{document}
	\maketitle
	
	
	In questa esperienza si utilizza un reticolo di diffrazione per analizzare le righe spettrali di lampade al Na e al Cd.  [Da qualche parte all'inizio va messa la formula del reticolo]
	
	Si intende utilizzare prima la lampada al Na per misurare il passo del reticolo avvalendosi di una riga spettrale con lunghezza d'onda $\lambda$ nota, per poi osservare lo spettro della lampada al Cd e ricavarne le lunghezze d'onda delle principali righe spettrali.
	
	L'apparato sperimentale, [mostrato in Figura 1], è costituito da un piattello rotante intorno a un asse fisso con un collimatore ed un cannocchiale ad esso solidali ma capaci di ruotare indipendentemente da esso. Sull'asse è inoltre posto un portaoggetti fisso, sul quale si colloca il reticolo di diffrazione.
	
	Sul piattello è affisso un goniometro con sensibilità $S=0.5^{\circ} / div$ e al cannocchiale è affisso un indice dotato di un nonio che, utilizzato in combinazione con il goniometro, rende possibile prendere misure con errore di sensibilità pari a $\Delta S = 1'$.
	
	[Qualcosa sul collocare la lampada attaccata al collimatore, stringere il piu possibile la fenditura prima di usare il reticolo, aspettare che la lampada si riscaldi e altri accorgimenti?]

	Per prima cosa, è necessario assicurarsi che il fascio collimato proveniente dalla lampada incida perpendicolarmente sul reticolo di diffrazione al fine di evitare grossolani errori sistematici. 
	
	
	Per fare ciò si vuole verificare che la figura di diffrazione generata risulti simmetrica rispetto al massimo centrale.
	Utilizzando la lampada al Na, si misura prima la posizione angolare del massimo centrale $\Theta_1 = 83^\circ 24' \pm 1'$[serve a qualcosa includere la misura di theta0 prima di aver messo il reticolo?].
	Scegliendo poi come riga spettrale di riferimento quella più esterna del caratteristico "doppietto del sodio", sono state misurate le posizioni dei massimi relativi ad essa all'ordine +2 e -2, risultati essere rispettivamente pari a $\Theta_s = 128^\circ 44' \pm 1'$ e $\Theta_d = 37^\circ 10' \pm 1'$. [serve precisare gli errori?]
	Si osserva che le loro distanze dal massimo centrale, rispettivamente $\theta_s = |\Theta_s - \Theta_1| = 45^\circ20'$ e $\theta_d = 46^\circ 14'$, differiscono tra loro per $\Delta\theta = |\theta_s - \theta_d| = 54'$, portando a ritenere che il reticolo di diffrazione non risulti sufficientemente ben allineato.
	Per abbassare tale differenza, si sposta il cannocchiale di $\Delta\theta/2$ per collocarlo nella posizione in cui ci si aspetterebbe di trovare la riga e si ruota il reticolo, spostando la riga fino a renderla allineata con il cannocchiale.
	
	Per assicurarsi che tale procedura abbia corretto il disallineamento del reticolo, si prendono nuovamente le misure di $\Theta_1 = 83^\circ 24'$[La stessa di prima, vero?Perché questo fatto?], $\Theta_s = 129^\circ 9'$ e $\Theta_d = 37^\circ 34'$, dalle quali risulta $\Delta\theta = 4'$[compatibile con zero se fai propagazione errori?], ritenuto accettabile ai fini dell'esperienza. 
	
	
	
\end{document}