\documentclass{article}

\title{Relazione 4: Misura del passo di un reticolo di diffrazione
	e di lunghezze d'onda}

\date{}

\begin{document}
	\maketitle
	
	
	In questa esperienza si utilizza un reticolo di diffrazione per analizzare le righe spettrali di lampade al Na e al Cd.  [Da qualche parte all'inizio va messa la formula del reticolo]
	
	Si intende utilizzare prima la lampada al Na per misurare il passo del reticolo avvalendosi di una riga spettrale con lunghezza d'onda $\lambda$ nota, per poi osservare lo spettro della lampada al Cd e ricavarne le lunghezze d'onda delle principali righe spettrali.
	
	L'apparato sperimentale, [mostrato in Figura 1], è costituito da un piattello rotante intorno a un asse fisso con un collimatore ed un cannocchiale ad esso solidali ma capaci di ruotare indipendentemente da esso. Sull'asse è inoltre posto un portaoggetti fisso, sul quale si pone il reticolo di diffrazione.
	
	Sul piattello è affisso un goniometro con sensibilità $S=0.5^{\circ} / div$ e al cannocchiale è affisso un indice dotato di un nonio che, utilizzato in combinazione con il goniometro, rende possibile prendere misure con errore di sensibilità pari a $\Delta S = 1'$.
	
	[Qualcosa sul collocare la lampada attaccata al collimatore, stringere il piu possibile la fenditura prima di usare il reticolo, aspettare che la lampada si riscaldi e altri accorgimenti?]

	Per prima cosa, è necessario assicurarsi che il fascio collimato proveniente dalla lampada incida perpendicolarmente sul reticolo di diffrazione al fine di evitare grossolani errori sistematici. 
	
	Al fine di misurare il passo $p$ del reticolo, si posiziona la lampada al sodio di fronte al collimatore e si misurano le posizioni angolari del massimo centrale $\Theta_0$ e dei massimi principali $\Theta_m$ fino all'ordine 2. Le distanze angolari dei massimi principali dal massimo centrale si ottengono, con i segni corretti, come $\theta_m = \Theta_m - \Theta_0$. In Tabella 2 sono riportati i valori di $m$, $\theta_m$, $\sin \theta_m$. Sul Grafico 1 è rappresentato l'andamento di $\sin \theta_m$ in funzione dell'ordine $m$. Tramite una regressione lineare è possibile stimare la pendenza della retta di best_fit $\sin \theta_m = Am + B$ e verificare che l'intercetta $B$ risulti compatibile con zero. Dunque, è possibile stimare il passo del reticolo come $p=\frac{\lambda}{\A}$. Dalla regressione lineare effettuata sui dati in Tabella 2, risultano $A=\pm$ e $B=\pm$, da cui $p=(\pm)\mu m$. Se ne conclude che i dati speramentali sono in accordo con la legge teorica del reticolo e che la stima di $p$ è consistente con il valore nominale fornito dal costruttore.

	Volendo ora determinare le lunghezze d'onda dello spettro di emissione del cadmio, si sostituisce la lampada al Na con una lampada al Cd e si ripete la misura delle posizioni angolari dei massimi principali (incluso il massimo centrale) per ciascun colore dello spettro visibile. In particolare si distinguono quattro colori: indaco, azzurro, verde e rosso. Nelle Tabelle 3 - 6 sono riportate le misure di $m$, $\theta_m$, $\sin \theta_m$ rispettivamente per ciascuno dei suddetti colori. Parimenti, nei Grafici 2 - 5 sono riportati gli andamenti di $\sin \theta_m$ in funzione dell'ordine $m$, sui quali si eseguono altrettante regressioni lineari. Di seguito si riportano i risultati (parametri $A$ e $B$) delle suddette regressioni lineari e la stima corrispondente della lunghezza d'onda, ottenuta come $\lambda_{colore}= A_{colore} p$, con il valore di $p$ precedentemente stimato.

	In conclusione, si osserva un buon accordo tra i dati sperimentali relativi a ciascun colore dello spettro visibile del cadmio con la legge teorica del reticolo. Inoltre, la stima di ciascuna lunghezza d'onda è compatibile con il valore atteso per essa.
\end{document}
