\documentclass{article}

\title{Relazione 4: Misura del passo di un reticolo di diffrazione
	e di lunghezze d'onda}

\date{}

\begin{document}
	\maketitle
	
	
	In questa esperienza si utilizza un reticolo di diffrazione per analizzare le righe spettrali di lampade al Na e al Cd. [Usando la relazione $\sin{\theta}=m\frac{\lambda}{p}$ che lega l'angolo $\theta$ dal massimo centrale al rapporto tra la lunghezza d'onda $\lambda$ e il passo del reticolo $p$] [Da qualche parte all'inizio va messa la formula del reticolo]
	
	Si intende utilizzare prima la lampada al Na per misurare il passo del reticolo avvalendosi di una riga spettrale con lunghezza d'onda $\lambda$ nota, per poi osservare lo spettro della lampada al Cd e ricavarne le lunghezze d'onda delle principali righe spettrali.
	
	L'apparato sperimentale, [mostrato in Figura 1], è costituito da un piattello rotante intorno a un asse fisso con un collimatore ed un cannocchiale ad esso solidali ma capaci di ruotare indipendentemente da esso. Sull'asse è inoltre posto un portaoggetti fisso, sul quale si pone il reticolo di diffrazione.
	
	Sul piattello è affisso un goniometro con sensibilità $S=0.5^{\circ} / div$ e al cannocchiale è affisso un indice dotato di un nonio che, utilizzato in combinazione con il goniometro, rende possibile prendere misure con errore di sensibilità pari a $\Delta S = 1'$.

 Dopo aver acceso la lampada al Na e averla posta di fronte al collimatore, è necessario aspettare qualche minuto per assicurarsi che abbia raggiunto la temperatura di lavoro. Poi si riduce al l'apertura della fenditura del collimatore e si osserva la sua immagine attraverso il cannocchiale: in questo modo ci si può assicurare che l'immagine sia nitida e abbia le minime dimensioni trasversali, ma anche che sia sufficientemente luminosa e visibile.

Prima di porre il reticolo sul piattello, si allinea il cannocchiale con il collimatore e se ne registra la posizione angolare $\Theta_0$. Si procede quindi a collocare il reticolo al centro del piattello, perpendicolarmente al fascio di luce proveniente dal collimatore. Prima di continuare, si ci assicura che il reticolo sia correttamente posizionato entro un errore di $\varepsilon = 5'$, al fine di evitare errori sistematici.


	
 [MATTEO]
 
 
Per fare ciò si vuole verificare che la figura di diffrazione generata risulti simmetrica rispetto al massimo centrale.
Utilizzando la lampada al Na, si misura prima la posizione angolare del massimo centrale $\Theta_1 = 83^\circ 24' \pm 1'$[serve a qualcosa includere la misura di theta0 prima di aver messo il reticolo?].
Scegliendo poi come riga spettrale di riferimento quella più esterna del caratteristico "doppietto del sodio", sono state misurate le posizioni dei massimi relativi ad essa all'ordine +2 e -2, risultati essere rispettivamente pari a $\Theta_s = 128^\circ 44' \pm 1'$ e $\Theta_d = 37^\circ 10' \pm 1'$. [serve precisare gli errori?]
Si osserva che le loro distanze dal massimo centrale, rispettivamente $\theta_s = |\Theta_s - \Theta_1| = 45^\circ20'$ e $\theta_d = 46^\circ 14'$, differiscono tra loro per $\Delta\theta = |\theta_s - \theta_d| = 54'$, portando a ritenere che il reticolo di diffrazione non risulti sufficientemente ben allineato.
Per abbassare tale differenza, si sposta il cannocchiale di $\Delta\theta/2$ per collocarlo nella posizione in cui ci si aspetterebbe di trovare la riga e si ruota il reticolo, spostando la riga fino a renderla allineata con il cannocchiale.

Per assicurarsi che tale procedura abbia corretto il disallineamento del reticolo, si prendono nuovamente le misure di $\Theta_1 = 83^\circ 24'$[La stessa di prima, vero?Perché questo fatto?], $\Theta_s = 129^\circ 9'$ e $\Theta_d = 37^\circ 34'$, dalle quali risulta $\Delta\theta = 4'$[compatibile con zero se fai propagazione errori?], ritenuto accettabile ai fini dell'esperienza. 


[VITTORIO]

Al fine di misurare il passo $p$ del reticolo, si posiziona la lampada al sodio di fronte al collimatore e si misurano le posizioni angolari del massimo centrale $\Theta_0$ e dei massimi principali $\Theta_m$ fino all'ordine 2. Le distanze angolari dei massimi principali dal massimo centrale si ottengono, con i segni corretti, come $\theta_m = \Theta_m - \Theta_0$. In Tabella 2 sono riportati i valori di $m$, $\theta_m$, $\sin \theta_m$. Sul Grafico 1 è rappresentato l'andamento di $\sin \theta_m$ in funzione dell'ordine $m$. Tramite una regressione lineare è possibile stimare la pendenza della retta di best_fit $\sin \theta_m = Am + B$ e verificare che l'intercetta $B$ risulti compatibile con zero. Dunque, è possibile stimare il passo del reticolo come $p=\frac{\lambda}{\A}$. Dalla regressione lineare effettuata sui dati in Tabella 2, risultano $A=\pm$ e $B=\pm$, da cui $p=(\pm)\mu m$. Se ne conclude che i dati speramentali sono in accordo con la legge teorica del reticolo e che la stima di $p$ è consistente con il valore nominale fornito dal costruttore.

Volendo ora determinare le lunghezze d'onda dello spettro di emissione del cadmio, si sostituisce la lampada al Na con una lampada al Cd e si ripete la misura delle posizioni angolari dei massimi principali (incluso il massimo centrale) per ciascun colore dello spettro visibile. In particolare si distinguono quattro colori: indaco, azzurro, verde e rosso. Nelle Tabelle 3 - 6 sono riportate le misure di $m$, $\theta_m$, $\sin \theta_m$ rispettivamente per ciascuno dei suddetti colori. Parimenti, nei Grafici 2 - 5 sono riportati gli andamenti di $\sin \theta_m$ in funzione dell'ordine $m$, sui quali si eseguono altrettante regressioni lineari. Di seguito si riportano i risultati (parametri $A$ e $B$) delle suddette regressioni lineari e la stima corrispondente della lunghezza d'onda, ottenuta come $\lambda_{colore}= A_{colore} p$, con il valore di $p$ precedentemente stimato.

In conclusione, si osserva un buon accordo tra i dati sperimentali relativi a ciascun colore dello spettro visibile del cadmio con la legge teorica del reticolo. Inoltre, la stima di ciascuna lunghezza d'onda è compatibile con il valore atteso per essa.




 
	
	
\end{document}
